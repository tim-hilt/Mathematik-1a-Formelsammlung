\message{ !name(Mathe1aFormelsammlung.tex)}\documentclass[8pt, a4paper]{article}

\usepackage[german]{babel}
\usepackage[utf8]{inputenc}
\usepackage{mathtools}

\title{Mathematik 1a Formelsammlung}
\author{Tim Hilt}
\date{24. März 2018}

\begin{document}

\message{ !name(Mathe1aFormelsammlung.tex) !offset(-3) }

\maketitle
\pagebreak

\tableofcontents
\pagebreak

\section{Grundlagen}
\subsection{Quadratische Ergänzung}

Eine quadratische Funktion
\[f(x)=ax^2+bx+c\]
kann geschrieben werden als
\[f(x)=a \left (x+\frac{b}{2a} \right)^2+ \left (c-\frac{b^2}{4a} \right)\]

\setcounter{section}{4}

\section{Funktionen}
\subsection{Schnittpunkte mit y-Achse}
Für Schnittpunkt mit der y-Achse \(f(x)\) mit \(x=0\) auflösen!

\subsection{Nullstellen}
Für Schnittpunkte mit der x-Achse \(f(x)=0\) setzen und Gleichung lösen!\\
Wenn Funktionsgleichung in hässlichem Format steht versuchen \(f(x)=0\) umzustellen und zu verschönern


\end{document}
%%% Local Variables:
%%% mode: latex
%%% TeX-master: t
%%% End:
\message{ !name(Mathe1aFormelsammlung.tex) !offset(-45) }
